\documentclass[10pt]{article}
\usepackage{amsmath, amssymb, amsthm}
\usepackage[top=2cm, left = 2cm, right = 2cm, bottom = 3cm]{geometry}
\usepackage[pdftex]{graphicx}
\usepackage{asymptote}
\usepackage{fancyhdr}
\newcommand{\N}{\mathbb{N}}
\pagestyle{fancy}
\rhead{}
\chead{\includegraphics[scale=0.1]{../CMIMC-header-2018.png}}
\lhead{}
\setlength{\headheight}{43pt}
\rfoot{}
\cfoot{}
\lfoot{}
\newcommand{\proposed}[1]
{
\vspace{5pt}
\noindent\textit{Proposed by #1}
}
\newcommand{\solution}
{
\vspace{5pt}
\noindent\textit{Solution.}\qquad
}
\begin{document}\thispagestyle{empty}
\begin{center}

\vspace*{40pt}

\includegraphics[scale=0.2]{../CMIMC-header-2018.png}

\includegraphics[scale=0.35]{combo-header.png}

\vspace{1.4in}

\includegraphics[scale=0.20]{Instruction-Header.png}
\noindent\rule{15.7cm}{2pt}
\end{center}

\vspace{10pt}

\input{input/test-instructions}

\vspace{0.7in}

\begin{center}
\includegraphics[scale=0.15]{../sponsor-footer.png}
\end{center}
\newpage

\begin{center}
\huge\textbf{Combinatorics}\normalsize

\vspace{3pt}
\end{center}

\begin{enumerate}

\item Ninety-eight apples who always lie and one banana who always tells the truth are randomly arranged along a line. The first fruit says ``One of the first forty fruit is the banana!'' The last fruit responds ``No, one of the \emph{last} forty fruit is the banana!'' The fruit in the middle yells ``I'm the banana!'' In how many positions could the banana be?

\item Compute the number of ways to rearrange nine white cubes and eighteen black cubes into a $3\times 3\times 3$ cube such that each $1\times1\times3$ row or column contains exactly one white cube. Note that rotations are considered \textit{distinct}.

\item Michelle is at the bottom-left corner of a $6\times 6$ lattice grid, at $(0,0)$. The grid also contains a pair of one-time-use teleportation devices at $(2,2)$ and $(3,3)$; the first time Michelle moves to one of these points she is instantly teleported to the other point and the devices disappear. If she can only move up or to the right in unit increments, in how many ways can she reach the point $(5,5)$?

\item At CMU, the A and the B buses arrive once every 20 and 18 minutes, respectively. Kevin prefers the A bus but does not want to wait for too long.  He commits to the following waiting scheme: he will take the first A bus that arrives, but after waiting for five minutes he will take the next bus that comes, no matter what it is. Determine the probability that he ends up on an A bus.

\item Victor shuffles a standard 54-card deck then flips over cards one at a time onto a pile stopping after the first ace. However, if he ever reveals a joker he discards the entire pile, including the joker, and starts a new pile; for example, if the sequence of cards is 2-3-Joker-A, the pile ends with one card in it. Find the expected number of cards in the end pile.

\item Richard rolls a fair six-sided die repeatedly until he rolls his twentieth prime number or his second even number. Compute the probability that his last roll is prime.

\item Nine distinct light bulbs are placed in a circle, each of which is off. Determine the number of ways to turn on some of the light bulbs in the circle such that no four consecutive bulbs are all off. %in each group of four consecutive light bulbs at least one of the bulbs is on.

\item Fred and George play a game, as follows. Initially, $x = 1$. Each turn, they pick $r \in \{3,5,8,9\}$ uniformly at random and multiply $x$ by $r$. If $x+1$ is a multiple of 13, Fred wins; if $x+3$ is a multiple of 13, George wins; otherwise, they repeat. Determine the probability that Fred wins the game.

\item Compute the number of rearrangements $a_1, a_2, \dots, a_{2018}$ of the sequence $1, 2, \dots, 2018$ such that $a_k > k$ for \textit{exactly} one value of $k$.

\item Call a set $S \subseteq \{0,1,\dots,14\}$ \textit{sparse} if $x+1 \pmod{15}$\footnote{here referring to the remainder when $x+1$ is divided by $15$} is not in $S$ whenever $x \in S$. Find the number of sparse sets $T$ such that the sum of the elements of $T$ is a multiple of 15.

\end{enumerate}

\end{document}
