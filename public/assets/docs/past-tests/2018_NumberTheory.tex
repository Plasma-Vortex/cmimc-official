\documentclass[10pt]{article}
\usepackage{amsmath, amssymb, amsthm}
\usepackage[top=2cm, left = 2cm, right = 2cm, bottom = 3cm]{geometry}
\usepackage[pdftex]{graphicx}
\usepackage{asymptote}
\usepackage{fancyhdr}
\newcommand{\N}{\mathbb{N}}
\pagestyle{fancy}
\rhead{}
\chead{\includegraphics[scale=0.1]{../CMIMC-header-2018.png}}
\lhead{}
\setlength{\headheight}{43pt}
\rfoot{}
\cfoot{}
\lfoot{}
\newcommand{\proposed}[1]
{
\vspace{5pt}
\noindent\textit{Proposed by #1}
}
\newcommand{\solution}
{
\vspace{5pt}
\noindent\textit{Solution.}\qquad
}
\begin{document}\thispagestyle{empty}
\begin{center}

\vspace*{40pt}

\includegraphics[scale=0.2]{../CMIMC-header-2018.png}

\includegraphics[scale=0.3]{nt-header.png}

\vspace{1.4in}

\includegraphics[scale=0.20]{Instruction-Header.png}
\noindent\rule{15.7cm}{2pt}
\end{center}

\vspace{10pt}

\input{input/test-instructions}

\vspace{0.7in}

\begin{center}
\includegraphics[scale=0.15]{../sponsor-footer.png}
\end{center}
\newpage

\begin{center}
\huge\textbf{Number Theory}\normalsize

\vspace{3pt}
\end{center}

\begin{enumerate}
\item Suppose $a$, $b$, and $c$ are relatively prime integers such that \[\frac{a}{b+c} = 2\qquad\text{and}\qquad \frac{b}{a+c} = 3.\] What is $|c|$?

\item Find all integers $n$ for which $(n-1)\cdot 2^n + 1$ is a perfect square.

\item Let $S$ be the set of natural numbers that cannot be written as the sum of three squares. Legendre's three-square theorem states that $S$ consists of precisely the integers of the form $4^a(8b+7)$ where $a$ and $b$ are nonnegative integers. Find the smallest $n\in\mathbb N$ such that $n$ and $n+1$ are both in $S$.

\item Let $a>1$ be a positive integer.  The sequence of natural numbers $\{a_n\}_{n\geq 1}$ is defined such that $a_1 = a$ and for all $n\geq 1$, $a_{n+1}$ is the largest prime factor of $a_n^2 - 1$.  Determine the smallest possible value of $a$ such that the numbers $a_1$, $a_2$,$\ldots$, $a_7$ are all distinct.

\item It is given that there exist unique integers $m_1,\ldots, m_{100}$ such that \[0\leq m_1 < m_2 < \cdots < m_{100}\quad\text{and}\quad 2018 = \binom{m_1}1 + \binom{m_2}2 + \cdots + \binom{m_{100}}{100}.\] Find $m_1 + m_2 + \cdots + m_{100}$.

\item Let $\phi(n)$ denote the number of positive integers less than or equal to $n$ that are coprime to $n$. Find the sum of all $1<n<100$ such that $\phi(n)\mid n$.

\item For each $q\in\mathbb Q$, let $\pi(q)$ denote the period of the repeating base-$16$ expansion of $q$, with the convention of $\pi(q)=0$ if $q$ has a terminating base-$16$ expansion. Find the maximum value among \[\pi\left(\frac11\right),~\pi\left(\frac12\right),~\dots,~\pi\left(\frac1{70}\right).\]

\item It is given that there exists a unique triple of positive primes $(p,q,r)$ such that $p<q<r$ and \[\dfrac{p^3+q^3+r^3}{p+q+r} = 249.\] Find $r$.

\item Let $\phi(n)$ denote the number of positive integers less than or equal to $n$ that are coprime to $n$. Compute \[\sum_{n=1}^{\infty}\frac{\phi(n)}{5^n+1}.\]

\item Let $a_1 < a_2 < \cdots < a_k$ denote the sequence of all positive integers between $1$ and $91$ which are relatively prime to $91$, and set $\omega = e^{2\pi i/91}$. Define \[S = \prod_{1\leq q < p\leq k}\left(\omega^{a_p} - \omega^{a_q}\right).\] Given that $S$ is a positive integer, compute the number of positive divisors of $S$.
\end{enumerate}

\end{document}
