\documentclass[10pt]{article}
\usepackage{amsmath, amssymb, amsthm}
\newcommand{\abs}[1]{\lvert #1 \rvert}
\newtheorem{lemma}{Lemma}
\usepackage[top=2cm, left = 2cm, right = 2cm, bottom = 3cm]{geometry}
\usepackage[pdftex]{graphicx}
\usepackage{asymptote}
\usepackage{fancyhdr}
\newcommand{\N}{\mathbb{N}}
\pagestyle{fancy}
\rhead{}
\chead{\includegraphics[scale=0.1]{../CMIMC-header-2018.png}}
\lhead{}
\setlength{\headheight}{43pt}
\rfoot{}
\cfoot{}
\lfoot{}
\newcommand{\proposed}[1]
{
\vspace{5pt}
\noindent\textit{Proposed by #1}
}
\newcommand{\solution}
{
\vspace{5pt}
\noindent\textit{Solution.}\qquad
}
\begin{document}

\begin{center}
\huge\textbf{Combinatorics Solutions Packet}\normalsize

\vspace{3pt}
\end{center}

\begin{enumerate}

\item Ninety-eight apples who always lie and one banana who always tells the truth are randomly arranged along a line. The first fruit says ``One of the first forty fruit is the banana!'' The last fruit responds ``Well, one of the \emph{last} forty fruit is the banana!'' The fruit in the middle yells ``I'm the banana!'' In how many positions could the banana be?

\proposed{Patrick Lin}

\solution Ignore the fruit in the middle, since every fruit is able to say they are the banana. We have valid scenarios if exactly one of the first and last fruit tells the truth, which yields two positions for the banana, or if both of them lie, which implies the banana is one of the middle 19 fruits. This yields $2 + 19 = \boxed{21}$ possible locations for the banana.


\item Compute the number of ways to rearrange nine white cubes and eighteen black cubes into a $3\times 3\times 3$ cube such that each $1\times1\times3$ row or column contains exactly one white cube. Note that rotations are considered \textit{distinct}.

\proposed{Zimu Xiang}

\solution Note that there must be 3 white cubes in each $3\times 3$ layer of the cube. There are $3! = 6$ ways to place them in the first layer, $2$ ways to place them in the second layer such that they don't overlap with the first layer, and 1 way to place them in the last layer. This gives $6 \cdot 2 \cdot 1 = \boxed{12}$ ways.


%\item Say a set $S \subseteq \{1,2,\dots, 1000\}$ is \textit{diffident} if for any two distinct elements in $S$, their positive difference is not a power of two. Determine the maximum number of elements that can be in any diffident set.
%\proposed{Patrick Lin}
%\solution 
%We claim the answer is $\boxed{334}$. To show that it's achievable, pick all the numbers that are 1 mod 3. Any pairwise difference is a multiple of 3 and hence certainly not a power of 2. To show that this is maximal, note that any set of size greater than 334 must contain two elements $i,j$ such that $0 < \abs{i - j} < 3$. Since both 1 and 2 are powers of 2, no larger set is diffident, as desired.
%

\item Michelle is at the bottom-left corner of a $6\times 6$ lattice grid, at $(0,0)$. The grid also contains a pair of one-time-use teleportation devices located at $(2,2)$ and $(3,3)$; the first time Michelle moves to one of these points she is instantly teleported to the other point, and the devices disappear. If she can only move up or to the right in unit increments, find the number of ways in which can she reach the point $(5,5)$.

\proposed{Patrick Lin}

\solution There are three ways Michelle can reach $(5,5)$; she can use the teleportation device at $(2,2)$, the device at $(3,3)$, or use neither. In the first case, there are $\binom{4}{2}^2 = 36$ paths. In the second case, there are $(\binom{6}{3} - 2\binom{4}{2})\binom{6}{3} = 160$ paths, since we must avoid landing on $(2,2)$ on the way to $(3,3)$. Finally, in the last case, by inclusion-exclusion there are $\binom{10}{5} - 2\binom{4}{2}\binom{6}{3}+\binom{4}{2}\binom{2}{1}\binom{4}{2} = 84$ ways. Together, there are $36+160+84=\boxed{280}$ paths to $(5,5)$.


\item At CMU, the A and the B buses arrive once every 20 and 18 minutes, respectively. Kevin prefers the A bus but doesn't want to wait for too long, so he commits to the following waiting scheme: he'll take the first A bus that arrives, but after waiting for five minutes he'll take the next bus that comes, no matter what it is. Determine the probability that he ends up on an A bus.

\proposed{Patrick Lin}

\solution The shaded area gives the probability that Kevin takes the A bus; if the B bus arrives within the first 5 minutes then we almost certainly take the A bus, except for the small bit on the lower-right that occurs when, say, the B bus comes at 1 and 19 minutes while the A bus would've come in 20 minutes. If the B bus does not arrive within the first 5 minutes, then we take the A bus if it comes first. The desired probability is $1-\frac12\cdot\frac{15^2}{20\cdot18} = \boxed{\tfrac{11}{16}}$.

\begin{center}
\includegraphics[scale=0.65]{C4}
\end{center}



\item Victor shuffles a standard 54-card deck, then flips over cards one at a time onto a pile, stopping after the first ace. However, if he ever reveals a joker he discards the entire pile, including the joker, and starts a new pile; for example, if the sequence of cards is 2-3-Joker-A, the pile ends with one card in it. Find the expected number of cards in the end pile.

\proposed{Patrick Lin}

\solution Call the four aces and the two jokers \textit{special} cards. Conditioned on the first ace being the $i$-th special card to appear, the number of piles is equal to the one plus the number of cards between the $i$-th and $(i-1)$-th special cards; by symmetry, this is one plus the expected number of cards until we draw the first special card, which is $1 + \frac{48}{7} = \boxed{\tfrac{55}{7}}$.



\item Richard rolls a fair six-sided die repeatedly until he rolls his twentieth prime number or his second even number. Compute the probability that his last roll is prime.

\proposed{Patrick Lin}

\solution We'll solve the more general case of rolling $k$ prime numbers versus two even numbers. We may assume without loss of generality that no 1 is ever rolled. Note that he is guaranteed to finish rolling after $k+1$ rolls, and that his last roll can be prime only if he finishes rolling after $k$ or $k+1$ rolls. In the former case, he can roll $k-1$ elements of $\{3,5\}$ followed by something in $\{2,3,5\}$, or one of the first $k-1$ rolls can be a 2. Together, there are $3\cdot 2^{k-1} + 3\cdot(k-1)\cdot 2^{k-2}$ possibilities. In the latter case, there must be $k-1$ rolls in $\{3,5\}$ and one roll in $\{4,6\}$ in the first $k$ rolls, and the last roll may be any prime. This gives $3k\cdot 2^k$ possibilities. The total probability is hence
\[\frac{5(3\cdot2^{k-1} + 3(k-1)\cdot2^{k-2}) + 3k\cdot2^k}{5^{k+1}} = \frac{2^{k-2}}{5^{k+1}}(27k+15).\]
Substituting $k = 20$ gives an answer of $\boxed{\tfrac{111\cdot2^{18}}{5^{20}}}$.



\item Nine distinct light bulbs are placed in a circle. Each light bulb can be on or off. In order to properly light up the room, in each group of four adjacent light bulbs, at least one must be turned on. How many such configurations are there?

\proposed{Andy Yang}

\solution Let $f(n)$ be the number of valid configurations of $n$ light bulbs in a line such that both the first and last bulbs are on. Clearly $f(x) = 0$ for $x < 1$ and $f(1) = 1$. Recursively, we have $f(n) = f(n-1) + f(n-2) + f(n-3) + f(n-4)$. Now, number the light bulbs in the circle from 1 through 9. In any valid configuration, consider the smallest and largest light bulbs that are turned on: they are distinct and cannot be more than 4 bulbs apart. Doing casework on the distance between them gives an answer of $1\cdot f(9) + 2\cdot f(8) + 3\cdot f(7) + 4\cdot f(6) = \boxed{367}$.



\item Fred and George play a game, as follows. Initially, $x = 1$. Each turn, they pick $r \in \{3,5,8,9\}$ uniformly at random and multiply $x$ by $r$. If $x+1$ is a multiple of 13, Fred wins; if $x+3$ is a multiple of 13, George wins; otherwise, they repeat. Determine the probability that Fred wins the game.

\proposed{Patrick Lin}

\solution Working modulo 13, observe that $3 = 9^{-1}$, $5 = 8^{-1}$, and $3^3 = 5^4 = 1$. We can then think of this problem as moving infinitely along the grid shown below, making a step either up, down, left, or right in each turn.

\begin{table}[!h]
\centering
\begin{tabular}{|cccc|}
\hline
1 & 5 & 12 & 8  \\
3 & 2 & 10 & 11 \\
9 & 6 & 4  & 7  \\ \hline
\end{tabular}
\end{table}

Let $f(n)$ be the probability that Fred wins, given that we're currently on $n$. Clearly $f(12) = 1$ and $f(10) = 0$; invoking symmetry, we have $f(9) = f(6) = f(4) = f(7) = \frac12$, $f(5) = f(8)$, and $f(2) = f(11)$. Furthermore, we have $f(3) = 1 - f(1)$ and $f(2) = 1 - f(5)$. Hence, this reduces to the system of equations
\begin{align*}
f(1) & = \frac14\left(f(9) + f(5) + f(3) + f(8)\right) = \frac14\left(\frac32 - f(1) + 2f(5)\right) = \frac{3}{10} + \frac25 f(5) \\
f(5) & = \frac14\left(f(6) + f(12) + f(2) + f(1)\right) = \frac14\left(\frac52 - f(5) + f(1)\right) = \frac12 + \frac15 f(1).
\end{align*}
Solving for $f(1)$ gives the answer of $\boxed{\tfrac{25}{46}}$.


\item Compute the number of rearrangements $a_1, a_2, \dots, a_{2018}$ of the sequence $1, 2, \dots, 2018$ such that $a_k > k$ for \textit{exactly} one value of $k$.

\proposed{David Altizio}

\solution The key to this problem is the following result.

\begin{lemma}
Consider the rearrangement as a permutation $\sigma:[2018]\to[2018]$.  Then $\sigma(k) > k$ for exactly one value of $k$ if and only if the cycle decomposition of $\sigma$ can be written in the form \[(b_1\,\,\,\,\,b_2\,\,\,\,\cdots\,\,\,\,b_\ell)\] where $b_1>b_2>\cdots>b_\ell$ are distinct integers between $1$ and $2018$.
\end{lemma}

\begin{proof}
Note that any such permutation clearly works; in particular, the unique $k$ satisfying the requested property is $k=b_\ell$, since then $\sigma(b_\ell) = b_1 > b_\ell$.  Now consider any permutation $\sigma$ satisfying the above property.  Remark that the cycle decomposition of $\sigma$ must contain only one cycle.  This is because every cycle contributes at least one value of $k$ for which $\sigma(k) > k$, since otherwise writing the elements in the cycle as $c_1,\ldots, c_m$ we would have \[c_1 > c_2 > \cdots > c_m > c_1,\] which is clearly not possible.  In particular, this cycle must be the one containing $k$.  Similar logic as above shows that the remaining elements of the cycle must be sorted in decreasing order, and so we are done.
\end{proof}

With this observation, the problem becomes easy.  Every such permutation bijects to a unique subset $S\subseteq [2018]$ with $|S|\geq 2$, since the permutation depends only on the set of elements which make up the cycle.  Thus, the requested answer is the number of subsets of $[2018]$ which have at least $2$ elements, or $\boxed{2^{2018} - 2019}$.

% if there's a good way to genfunc this one let me know
\item Call a subset $S \subseteq \{0,1,\dots,14\}$ \textit{sparse} if $x+1 \pmod{15}$ is not in $S$ whenever $x \in S$. Find the number of sparse subsets such that the sum of their elements is a multiple of 15.

\proposed{Patrick Lin}

\solution First, we'll give a sketch of the following claim: for odd $n$, the number of subsets of $\{1,\dots,n\}$ that sum to a multiple of $n$ is equal to the number of necklaces on $n$ beads, where each bead is black or white. If $n$ is prime, then the natural transformation shows a bijection: given a subset $S$, label the beads from $1$ to $n$ and set the $i$-th bead white if $i \in S$. If $0 < \abs{S} < n$, then $\abs{S}$ and $n$ are coprime, and so exactly one cyclic shift of $S$ has sum equal to a multiple of $n$. If $\abs{S} = 0$ or $\abs{S} = n$, every cyclic shift is itself, and so this is also true.

\par Now, if $n$ is composite and $\abs{S} = d \mid n$, then there are several subsets that might map to the same necklace. Fortunately, there are $d$ cyclic shifts with the same remainder modulo $n$, and the set of all cyclic shifts covers $\tfrac{n}{d}$ values. It is then possible to slightly alter each of these $d$ cyclic shifts to cover all remainders, after which the same argument as above will work.

\par It remains to compute the number of necklaces on $n$ beads that don't have two adjacent white beads. Let $f(m)$ denote the number of ways do so with $m$ beads, counting rotations as \textit{distinct} objects, and let $g(m)$ denote the number of ways to do so with $m$ beads in a straight line. It's well known (and comes from the recurrence $g(m) = g(m-1) + g(m-2)$) that $g(m) = F_{m+1}$, where $F_k$ denotes the $k$-th Fibonacci number. To compute $f$, we simply note that we need the same condition in $g$ but also that at least one of the endpoints is a 0. This gives $f(m) = g(m) - g(m - 2) = F_{n+1} - F_{n-1}$. Finally, applying Burnside's lemma gives an answer of $\frac{1}{n} \sum_{d \mid n} \varphi(d) f(\frac{n}{d})$.

\par Substituting $n = 15$ yields a final answer of \[\frac{1\cdot f(15) + 2\cdot f(5) + 4\cdot f(3) + 8\cdot f(1)}{15} = \boxed{94}.\]

\end{enumerate}

\end{document}
